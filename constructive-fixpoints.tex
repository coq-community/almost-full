\documentclass{llncs} 
%\usepackage{MnSymbol}
\usepackage{amsmath, amssymb,amsfonts,mathrsfs,stmaryrd}
\usepackage{bussproofs,xspace}
\usepackage[x11names]{xcolor}
\usepackage{multicol}
\usepackage{listings}
\lstdefinelanguage{coq}{
xleftmargin=0.5em,
basicstyle=\small\ttfamily,
morekeywords={Record,Axiom,Implicit,Arguments,Notation,Definition,end,Fixpoint,CoFixpoint,forall,fun,Inductive,CoInductive,Lemma,match,Program,
Require,Import,Set,Structure,Canonical,
Prop,Type,with,Theorem,Proof,Qed,Defined,Corollary,Ltac},
%otherkeywords={->},
morecomment=[l]{//},morecomment=[s]{(*}{*)},morestring=[b]",
keywordstyle=\color{blue},
commentstyle=\color{Green4},
stringstyle=\color{red},
showstringspaces=false,
rangeprefix=\(\*\=,
rangesuffix=\ \*\),
includerangemarker=false,
literate= {forall}{{${\forall}\;$}}1  {exists}{{${\exists}\;$}}1 {=>}{{${\Rightarrow}\;$}}1 {->}{{$\to\;$}}1 {<->}{{$\leftrightarrow\;$}}1 
{/\\}{{$\land\;$}}2
{\\/}{{$\lor\;$}}2
}
\lstloadlanguages{Haskell}
\lstnewenvironment{code}
    {\lstset{}%
      \csname lst@SetFirstLabel\endcsname}
    {\csname lst@SaveFirstLabel\endcsname}
    \lstset{
      basicstyle=\small\ttfamily,
      flexiblecolumns=false,
      basewidth={0.5em,0.45em},
morecomment=[n]{\{-}{-\}},
      literate={\\\\}{{\char`\\\char`\\}}1
               {->}{{$\to$}}2 {>=}{{$\geq$}}2 {<-}{{$\gets$}}2
               {<=}{{$\leq$}}2 {=>}{{$\Rightarrow$}}2 
               {\ .}{{$\circ$}}2 {\ .\ }{{$\circ$}}2
               {>>}{{>>}}2 {>>=}{{>>=}}2
               {|}{{$\mid$}}1               
% {+}{{$+$}}1 {/}{{$/$}}1 {*}{{$*$}}1 {=}{{$=$}}1
%               {>}{{$>$}}1 {<}{{$<$}}1 {\\}{{$\lambda$}}1              
    }

\lstdefinelanguage{hs}{
xleftmargin=0.5em,
columns=fixed,
basicstyle=\ttfamily\footnotesize,
morekeywords=[1]{data,type,where,let,in,module,case,of,deriving,if,then,else},
keywordstyle=[1]\color{Magenta4},
morecomment=[l]{--},
morecomment=[n]{\{-}{-\}},
alsoletter={|-,|,=,+,::,\$,==,+>,`div`,`seqI`},
morekeywords=[2]{|-,|,=,+,::,\$,==,+>,`div`,`seqI`},
keywordstyle=[2]\color{Yellow4},
morestring=[b]",
commentstyle=\color{Red4},
stringstyle=\color{Brown2},
showstringspaces=false,
rangeprefix=\-\-\ /,
rangesuffix=/,
includerangemarker=false,
literate={forall}{{${\forall}\;$}}1
         {exists}{{${\exists}\;$}}1
         {|-}{{${\vdash}\;$}}1
         {||-}{{${\vdash}\;$}}1
         {\\\\}{{\char`\\\char`\\}}1
         {\\}{{$\color{Yellow4}\lambda$}}1              
         {->}{{$\color{Yellow4}\to$}}2 {>=}{{$\color{Yellow4}\geqslant$}}2 {<-}{{$\color{Yellow4}\gets$}}2
         {<=}{{$\color{Yellow4}\leqslant$}}2 {=>}{{$\color{Yellow4}\Rightarrow$}}2 
         {\ .}{{$\color{Yellow4}\circ$}}2 {\ .\ }{{$\color{Yellow4}\circ$}}2
         {>>}{{>>}}2 {>>=}{{>>=}}2
%         {|}{{$\color{Yellow4}\mid$}}1               
%
%literate= {forall}{{${\forall}\;$}}1 {exists}{{${\exists}\;$}}1 {=>}{{${\color{Yellow4}\Rightarrow}\;$}}1 {->}{{$\color{Yellow4}\to\;$}}1 {<->}{{$\leftrightarrow\;$}}1 
}

\lstset{language=coq}
\lstset{columns=flexible}
\lstset{moredelim=[il][\color{red}]{`}}
%\lstset{moredelim=[il][\color{Cyan4}]{@}}
\lstnewenvironment{coq}{\lstset{columns=flexible}}{}

\newcommand{\inputcoq}[2]{\lstinputlisting[language=coq,linerange={#2}-End]{code/prop-bounded/#1.v}}
\newcommand{\inpuths}[2]{\lstinputlisting[language=hs,linerange={#2}-End]{code/#1.hs}\vspace{-4pt}}
\newcommand{\inputhsmath}[2]{\lstinputlisting[mathescape,language=hs,linerange={#2}-End]{code/#1.hs}\vspace{-4pt}}
\newcommand{\inputhaskell}[2]{\lstinputlisting[language=Haskell,linerange={#2}-End]{code/#1.hs}}




\usepackage{mdwlist}
\usepackage{float}
\usepackage{url}
\usepackage{prooftree}

%\bibliographystyle{plain}
\setlength{\parskip}{0.35\baselineskip plus 0.2\baselineskip minus 0.1\baselineskip}
\setlength{\parsep}{\parskip}
\setlength{\topsep}{0cm}
\setlength{\parindent}{0cm}

\usepackage{pgf}
\usepackage{tikz}
\floatstyle{boxed}
\restylefloat{figure}

\usepgflibrary{shapes.geometric}
\usepgflibrary{shapes.misc}
\usetikzlibrary{shapes.geometric}
\usetikzlibrary{shapes.misc}

%% \newtheorem{lemma}{Lemma}
%% \newtheorem{corollary}{Corollary}
%% \newtheorem{proposition}{Proposition}
%% \newtheorem{theorem}{Theorem}

%% \theoremstyle{definition}
%% \newtheorem{example}{Example}

%% \newtheorem{definition}{Definition}
%% \newenvironment{proof}[1][Proof]{\begin{trivlist}
%% \item[\hskip \labelsep {\bfseries #1}]}{\hfill$\Box$\end{trivlist}}


\newcommand{\dv}[1]{{\color{red}{\bf DV:} #1}} 

%% \renewcommand{\dv}[1]{}

\lstset{morecomment=[l][\color{Green4}]{>}}


\newcommand{\finpath}{\rightsquigarrow} 

\newcommand{\todo}[1]{{\color{red}{#1}}}

\bibliographystyle{plain}
\begin{document}

%% \conferenceinfo{ICFP'10,} {September 27--29, 2010, Baltimore, Maryland, USA.}
%% \CopyrightYear{2010}
%% \copyrightdata{978-1-60558-794-3/10/09} 


\title{Stop when you are Almost-Full}
\subtitle{Adventures in constructive termination}

%% \authorinfo{}{}{}
\author{Dimitrios Vytiniotis\inst{1} and Thierry Coquand\inst{2}}
\institute{Microsoft Research, Cambridge \\
\email{dimitris@microsoft.com}
\and
University of Gothenburg \\
\email{coquand@chalmers.se}}

\maketitle

\begin{abstract}

Disjunctive well-foundedness (used in Terminator), size-change 
termination, and well-quasi-orders (used in supercompilation and 
term-rewrite systems) are examples of techniques that have been 
successfully applied to automatic proofs of program termination and online termination
testing, respectively. Although these works originate in different
communities, there is an intimate connection between them -- they rely
on closely related principles and both employ similar arguments from
Ramsey theory. At the same time there is a notable absence of these
techniques in programming systems based on constructive type
theory. In this paper we'd like to highlight the aforementioned
connection and make the core ideas widely accessible to theoreticians
and Coq programmers, by offering a Coq development which culminates in
some novel tools for performing induction. The benefit is nice
composability properties of termination arguments at the cost of
intuitive and lightweight user obligations. Inevitably, we have to
present some Ramsey-like arguments: Though similar proofs are
typically classical, we offer an entirely constructive development
standing on the shoulders of Veldman and Bezem, and Richman and Stolzenberg.

%% online termination testing with well-quasi-orders have successfully
%% been applied to automatic termination 

%% In the recent years the so-called {\em disjunctive invariants} method
%% has proven to be an extremely successful approach to automatic proofs
%% of program termination, leading to industrial-strength tools such as
%% Terminator. {\em Size-change} termination is another (related) extremely
%% effective methodology for automatic program termination. In the core of both works lies an argument from 
%% Ramsey theory.

%% Furthermore, research on online termination testing and
%% supercompilation has for a while been using termination testing
%% criteria for function inlining based on {\em well-quasi-orders}, often
%% employing Ramsey-like arguments to form more complex termination
%% testing criteria from simpler ones.

%% There is an intimate connection between these worlds, and a notable absense
%% of similar techniques in Coq. In this article we'd like to highlight that
%% connection and make the core ideas widely accessible to theoreticians and 
%% Coq programmers, by offering a Coq development which culminates in some new 
%% variations of induction principles. Inevitably, we have to present some 
%% Ramsey-like arguments: Though similar proofs are typically classical, we 
%% offer an entirely constructive development standing on the shoulders of Veldman 
%% and Bezem, and Coquand.

\end{abstract}


\newcommand{\qa}[2]{\\[-14pt]{\color{Green4} \item {#1}} {\color{blue} #2}}
\newcommand{\qanocr}[2]{\color{Green4}{#1} \color{blue}{#2}}
\newcommand{\bools}{\mathbb{B}} 
\newcommand{\nats}{\mathbb{N}} 
\newcommand{\ints}{\mathbb{Z}}
\newcommand{\finints}[1]{\mathbb{Z}_{#1}}
\newcommand{\intrange}[2]{\{{#1}..{#2}\}}
\newcommand{\intsge}[1]{\{{#1}..\}}
\newcommand{\unitty}{\mathbf{1}}
\newcommand{\voidty}{\mathbf{0}}
\newcommand{\rats}{\mathbb{Q}}
\newcommand{\enc}{\mathit{enc}} \newcommand{\dec}{\mathit{dec}}
\newcommand{\app}{\oplus}
\newcommand{\pref}{\sqsubseteq}
\newcommand{\map}{\mathit{map}}
\newcommand{\inv}{\mathit{inv}}
\newcommand{\id}{\mathit{id}}

%% \category{D.1.1}{Programming Techniques}   
%%                 {Applicative (Functional) Programming}
%% \category{D.3.3}{Programming Languages}
%%                 {Language Constructs and Features} 
%% \category{E.4}{CODING AND INFORMATION THEORY}
%%               [Data compaction and compression]

%% \terms
%% Design, Languages, Theory

%\keywords
%games, encoders, decoders 

 
\section{Introduction}\label{s:intro}


\dv{Must make shorter and tighten}
Program termination has always been an exciting subject among researchers, dating 
back to the early days of computing. The reason is because program termination is at 
the same time {\em important} for software reliability, and {\em difficult} for general 
classes of programs. Despite the difficulties, however, several research communities 
have managed to make good progress.

Over the recent years, the so-called {\em transition invariants}~\cite{podelski-rybalchenko:transition}
method has been an extremely successful approach for automatic 
proofs of program termination, leading to industrial-strength
tools, such as Terminator~\cite{terminator}. {\em Size-change 
termination} (SCT) ~\cite{Lee+:sct,jones-bohr,Sereni} is another very successful methodology
for automatic proofs of program termination. 
In the core of both works lies a formal argument from Ramsey theory~\cite{Heizmann+:sct-disjinv}.

Furthermore, research on online termination testing~\cite{leuschel-wqos} and 
supercompilation~\cite{Sorensen95analgorithm} has for a while been using termination 
testing criteria for function reductions and inlining based on {\em well-quasi-orders}, often 
employing Ramsey-like arguments to form more complex termination testing criteria from simpler ones. 

There is an intimate connection between these worlds, and a notable absence of similar techniques
to help programmers {\em prove} the termination or totality of their fixpoint definitions in Coq and -- more 
generally -- in systems based on constructive type theory. To quote some recent work on size-change termination
for Isabelle~\cite{Krauss07}:
\begin{quote}
{\em ``Our proof uses classical logic, including the (infinite, but countable) axiom of choice. It would be interesting to investigate if the proof can be modified to work in a weaker framework''}
\end{quote}

In this paper we'd like to highlight the connection between these previous works and make the core ideas widely 
accessible to theoreticians and Coq programmers, by offering 
a Coq development which introduces some novel variations of induction principles. Inevitably, we have 
to present some Ramsey-like arguments: Though similar proofs are typically classical, we offer a 
constructive development in the footsteps of Veldman and Bezem~\cite{Veldman01041993,coquand-short},
and Richman and Stolzenberg~\cite{wqo-sets}.

Specifically, our contributions with this paper are:
\begin{itemize}
  \item We introduce a novel mechanism for type-based termination, 
        that of {\em almost-full} relations (Section~\ref{sect:af}), which is a weaker
        version of the more traditional well-quasi-orders, originating in intuitionistic mathematics.

  \item We formally explain the connection between almost-full relations and 
        well-founded relations (Section~\ref{sect:wf-to-af}), and prove a new induction 
        theorem based on almost-full relations. (Section~\ref{sect:af-to-wf})

  \item We demonstrate that, unlike well-founded relations, almost-full relations 
        compose nicely to form other almost-full relations (Section~\ref{sect:af-constructions}).
        In this context, of particular interest is a construction which constitutes 
        a contribution on its own: a short proof of (the constructive version 
        of) Ramsey's theorem (for binary relations). We give examples of composing almost-full
        relations to show termination. Thanks to the composability of almost-full relations, the
        user obligations from the new induction principles typically involve intuitive (and amenable to automation) 
        relation inclusion lemmas instead of proofs about accessibility predicates. 

  \item We can use our method to show examples from SCT that go beyond 
        simple lexicographic orders (Section~\ref{sect:af-in-practice}). We show that the SCT
        principle can be intuitionistically proved from our induction principle.

%%        \dv{Maybe connect the discussion paragraph about SCT to this example.} 

%% an extension of our induction principle (called AF power-induction) that allows 
%%         us to show more complicated examples, closer in spirit to {\em size-change} termination, 
%%         for which simple lexicographic orders just do not work. (Section~\ref{sect:af-in-practice}).

  \item Connecting our work with further work on automatic termination proofs, we 
        show how the almost-full induction principle can be instantiated to derive the 
        Terminator rule (which is based on the so-called {\em disjunctive well-foundedness}).
        (Section~\ref{sect:terminator}) 

  \item We generalize our proof of Ramsey's theorem to relations of transfinite arities, and offer an elegant and simpler proof of this theorem
        than older attempts \cite{Coquand:1994:ART:185268.185270,Fridlender97higmanslemma}. We show how our simpler statement of Ramsey's theorem
        for binary relations follows from the generalized version. We also carefully explain the differences compared to related work 
       (Section~\ref{sect:generalized-irt}) \dv{Must add section!}

 \item We briefly discuss mutual induction principles. (Section~\ref{sect:mutual})

\end{itemize}

We finally present related work (Section~\ref{sect:related}), and outline 
further directions for research. (Section~\ref{sect:future}).

Our accompanying development does not make use of any ``non-standard'' axioms 
in Coq (such as classical facts, proof-irrelevance, or even the more benign functional extensionality). 
It builds under Coq 8.3pl2. 
%% \begin{center}\footnotesize
%%    \url{http://research.microsoft.com/people/dimitris/cfp.tgz}
%% \end{center}

The new induction principles proposed in this paper
are not-necessarily more expressive or easier to use than 
other (particularly recent~\cite{chargueraud-10-fix,kraus+:fix,sozeau-rewrite}) related work -- this 
is a topic that deserves further investigation, engineering, and potentially automation 
support. On the other hand,  the induction principles that we propose here are 
quite amenable to the same automation that made Terminator and SCT successful and quite pleasant to use due to 
the composability of almost-full relations and the nature of the user obligations that arise. Apart from contributing to Coq's large arsenal
of 
techniques for recursion~\cite{coqart,bove-capretta,Megacz:coinductive-monad,sozeau-rewrite,chargueraud-10-fix}, the 
other significant contribution of this article is to bring together ideas from different research communities 
in a type-theoretic framework. 

\section{Well-quasi-orders and almost-full relations}\label{sect:af}

The starting point for this exploration will be {\em online termination testing}.
Online termination testing is concerned with the following problem: Assume that we monitor the execution 
of a program by means of observing the state or the arguments passed down in recursive calls; can we raise
an error as soon as we detect that the program might be diverging? The requirement for a sound termination tester
is that we {\em must} raise an error if the program is indeed divergent. Conversely, we can't expect in general to raise 
an error {\em only} if the program is divergent but we'd like our online termination tester to be as lenient as possible.

Assume now that the observed state of a program forms a sequence of values $s_1,s_2,\ldots$ -- in effect we'd 
like to detect if that sequence $s$ could be infinite. There is a very natural mathematical definition that
can help us here, and that is the notion of a {\em well-quasi-order} (WQO):

\vspace{5pt}\begin{definition}[Well-quasi-order]
For some set \lstinline|X|, a binary relation $\preceq$ of type \lstinline|X -> X -> Prop| is a well quasi order 
if (i) it is transitive and (ii) for every infinite sequence $s$ of elements of \lstinline|X| it is the 
case that there exist $i$ and $j$ with $i < j$ such that $s_i \preceq s_j$. 
\end{definition}

Why is this helpful? Consider the following online termination tester which accepts a user-provided 
WQO $\preceq$ as input: We will keep a record of all previous values we have observed and every time a new 
value $s_{new}$ appears, we will check if for some old value $s_{old}$ it is $s_{old} \preceq s_{new}$. If this is true then we will 
raise an error, otherwise we will record $s_{new}$ in our history and wait for the next value. Now, 
if the sequence was infinite then we definitely know that we will raise an error at some point (because
$\preceq$ is a WQO). Of course, conservatively, we might raise an error even when the sequence is not infinite
because the WQO provided was too conservative. 

Let us demonstrate this with an example. On the type \lstinline|nat|, the relation $\leq$ (\lstinline|le| in Coq) 
is a WQO: Think of any infinite sequence of natural numbers -- at some point we will meet a natural number 
which is greater or equal than some previous one. Hence, our online termination tester would raise an error 
for the following sequence: 
\[ 10,7,6,4,1,5,4,3,3,3,\ldots \] 
as soon as it encountered the element $5$, since $4 \leq 5$. Of course, if the sequence 
is actually finite (e.g. it ends after a hundred $3$ values) then, too bad: our WQO 
has been too conservative and we should have used a more lenient one.

The merits of WQOs for online termination testing have been discussed in
previous work~\cite{leuschel-wqos} so we will not go into 
details here. Their main advantage is that they can form extremely lenient termination 
tests by combining simpler WQOs (at the cost of having to record big portions of history).


\subsection{Almost-full relations}\label{ssect:af}

The mechanism described above is by now well-established for online termination 
testing, so it is quite natural to ask how it would look in type theory and check
if it can be used to {\em prove} termination, in addition to testing for termination.

Surprisingly, it turns out that a certain kind of relations that satisfy property (ii) in the 
definition of WQOs have been proposed by mathematicians in an entirely different 
domain: the development of an intuitionistic version of Ramsey 
theory~\cite{Veldman01041993}. These are the {\em almost-full} (AF) relations (term coined by Wim Veldman), 
and for the rest of this paper we will focus on {\em binary} AF relations.

An AF relation is an {\em inductive} characterization of relations that satisfy 
property (ii) above: 
\inputcoq{AlmostFull}{AF}
Assume that we know that a relation \lstinline|R| satisfies 
\lstinline|almost_full R|. If the proof object is built from the 
\lstinline|AF_ZT| constructor then we know that any two elements in an infinite sequence they will be related. If on the other hand the proof object is
built from the \lstinline|AF_SUP| constructor then, if we receive a first
element \lstinline|x| in a sequence then we know that 
\lstinline|almost_full (fun y z => R y z \/ R x y)|. 
This means that in any infinite
sequence that starts with \lstinline|x|, either there exist two elements in the rest of the sequence related by \lstinline|R| itself, or
some element which is related to the first element \lstinline|x|.

%% \inputcoq{AlmostFull}{WFT}

%% An instructive way to understand a well-founded tree is as a set of winning strategies in a game. At each point either 
%% the game has finished and we won (\lstinline|ZT|) or (in the case of \lstinline|SUP|) the opponent can provide a next 
%% value of type \lstinline|X| and we are asked to come up with a next move. Because the tree is inductive, all 
%% such ``moves'' end up in a winning position. 

%% But, the winning strategies of which game exactly does a well-founded tree represent? Well, we wrote before 
%% that we are interested in relations for which every infinite sequence contains two values that are related. 
%% Let us define when a well-founded tree of type \lstinline|WFT X| ``secures'' a binary 
%% relation \lstinline|A : X -> X -> Prop|: 
%% \inputcoq{AlmostFull}{SecureBy}
%% Now if we know that \lstinline|SecureBy R p| and the tree \lstinline|p| is \lstinline|ZT| then all 
%% elements of the domain \lstinline|X| are related, and hence the tree is a witness that 
%% every infinite sequence has two related elements. If the tree is \lstinline|SUP p| then, when presented with 
%% a new value \lstinline|x|, either \lstinline|x| {\em itself} is related to some future element \lstinline|y|, or we have
%% to keep going to find the two related elements in the future. This explains the relation \lstinline|fun y z => A y z \/ A x y|
%% used in the recursive call. Because \lstinline|p| is inductive, we cannot go on for ever. In the leaves we will have established 
%% that some values along the way were related by \lstinline|A|!

%% This leads to the definition of an AF relation, namely one that is ``secured'' in the way 
%% we have defined by some well-founded tree:
%% \inputcoq{AlmostFull}{AlmostFull}

Based on these intuitions we may prove condition (ii) of the definition of
WQOs: 
%% We can now prove condition (ii) of the definition of WQOs: 
%% \inputcoq{AlmostFull}{InfiniteChain}
%% The sketch of the proof is based on the discussion above. We use function \lstinline|f| as 
%% an infinite sequence. Imagine a cursor at point \lstinline|k| of the infinite sequence. Then 
%% we have two cases: Either the current tree is \lstinline|ZT| in which case we can simply
%% take the elements at positions \lstinline|k| and \lstinline|k+1|, or the tree is \lstinline|SUP p|. 
%% In the latter case, by induction, we are guaranteed to find two related elements in the future, 
%% or an element which is related to the element in position \lstinline|k|. In both cases we are done!
%% As a corollary: 
\inputcoq{AlmostFull}{InfiniteChainCorollary}
It is interesting to observe that corollary \lstinline|af_inf_chain| is quite analogous to 
the ``no infinite descending chain'' property which can be proved intuitionistically 
from Coq's inductive definition of well-founded relations (based on accessibility predicates)~\cite{coqart}.
The converse of \lstinline|af_inf_chain| holds classically but Veldman and Bezem~\cite{Veldman01041993} show 
that there exists a recursive counterexample and so does not hold in type theory. \dv{Thierry, is this true? One reviewer asked about this.}
This makes it impossible to use property (ii) as the very defining property of 
AF relations (instead of our inductive characterization) because that alternative definition
cannot be used for induction (see Theorem~\lstinline|wf_from_af| in Section~\ref{sect:af-to-wf}).

Notice also that we've stopped worrying about the transitivity condition -- indeed we are not going to need it!
Some of the proofs in later parts of the paper would be simpler 
(and they are, in related work~\cite{OnWellQuasiOrderingFiniteTrees}) but essentially 
all interesting properties of WQOs can be proved on AFs without requiring transitivity. \dv{What to do about this paragraph?}

\section{Well-founded vs. almost-full relations}\label{sect:wf-to-af}

To build up some more intuitions about AF relations, we now turn to the connection between
AF relations and well-founded relations. A well-founded relation can be constructively 
characterized as a relation where every element in its domain is {\em accessible}. 
The corresponding (standard) Coq definitions are: 
\begin{lstlisting}
Inductive Acc (A:Type) (R:A->A->Prop) (x:A) : Prop :=
  Acc_intro : (forall y : A, R y x -> Acc R y) -> Acc R x.
Definition well_founded :=
  fun (A:Type) (R:A -> A -> Prop) => forall a:A, Acc R a.
\end{lstlisting}
Coq comes with a library for constructing well-founded relations
as well as proofs that several relations on commonly used datatypes are well-founded, 
the $<$ relation on \lstinline|nat| being the simplest example. 

It is easy to construct AF relations from {\em decidable} well-founded relations:
If we are given a decidable WF relation \lstinline|R| then we will
show next that the relation \lstinline|fun x y => not (R y x)| is AF. This will enable 
us to re-use Coq libraries and lemmas for WF relations in developments for AF relations.
We will use an auxiliary lemma \lstinline|af_iter|, which proves by 
induction on an accesibility predicate for an element \lstinline|x| 
that a larger relation \lstinline|fun y z => not (R y x) \/ not (R z y)| is 
AF. 
\inputcoq{AlmostFull}{Decidable}
\inputcoq{AlmostFull}{AfTreeIter}
We accept as input an \lstinline|x| which is accessible and we induct on the accessibility predicate. We 
create an \lstinline|AF_SUP| object. When we are presented with a ``next'' 
element \lstinline|y| we check whether \lstinline|R y x| or not using the decidability predicate \lstinline|decR|. If not, then we have immediately found both the elements and we can return \lstinline|ZT|. If on the other hand
it is \lstinline|R y x| then we use the induction hypothesis.
\dv{I feel this informal sketch is unhelpful. 
Either make it more precise or drop.}

With this definition, it is not difficult to derive the following.
%\inputcoq{AlmostFull}{AfTree}
%\inputcoq{AlmostFull}{AfFromWf}
\inputcoq{AlmostFull}{AfFromWfCor}
Corollary \lstinline|af_from_wf| allows us to go from a 
decidable well-founded relation to an AF. With this principle, and taking into account 
that $<$ is decidable and a total order, we can actually prove:
\inputcoq{AFConstructions}{LeqAF}
since $\forall x y, x \leq y \leftrightarrow \lnot (y < x)$ on natural numbers.

\subsection{From almost-full to well-founded relations}\label{sect:af-to-wf}

So far AF relations appear to be a funny flipped-over version of Coq's WF relations, so 
it's time we saw how can they be used to prove termination. The key intuition comes
from online termination testing with WQOs. Recall that a WQO-based termination
checker takes a WQO $\preceq$ and a ``history'' of past values and when presented with a new 
value checks whether some old value from the history is related to this new value. 

Think now of the relation \lstinline|T : X -> X -> Prop| which relates all adjacent 
values $s_{i+1}$ and $s_{i}$ (and only those) in the input sequence. This is often called the {\em transition relation} 
of the program that generates this sequence. As a convention we will be using the first argument of \lstinline|T| as
the ``next'' value and the second as the ``current'' value (so that we have $T\;s_{i+1}\;s_i$ for every $i$). 
The termination test that our WQO-based checker effectively implements is that:
\[ T^{+} \cap  (\preceq)^{-1} = \emptyset \]
where $T^{+}$ is the transitive closure of $T$ and $(\preceq)^{-1}$ is just the inverse of $\preceq$. No infinite 
sequence can pass this test, because an infinite sequence will necessarily have elements related by $\preceq$! 
Put it another way, if the test succeeds the transition relation cannot have infinite chains -- well, it 
is well-founded!

Generalizing our intuition from transition relations to arbitrary relations, and weakening the 
assumptions from WQOs to AF relations, the following lemma is the most important result of this 
paper, hence we put it in a big box:
\begin{center}
\framebox{
\inputcoq{AlmostFull}{WfFromAf}
}
\end{center}
In the \lstinline|wf_from_af| lemma, \lstinline|clos_trans_1n X T| is just the 
transitive closure of \lstinline|T|, as defined in Coq's standard library. 

Using \lstinline|wf_from_af| we can derive a simple lemma for transitive AFs, which
are WQOs: 
\inputcoq{AlmostFull}{WfFromWqo}
For instance, for the $\leq$ relation on natural numbers, it is clearly the case 
that $\lambda x y. x \leq y \land \lnot(y \leq x)$ is WF. This relation is simply $<$.

Of course the proof of \lstinline|wf_from_af| is done in a constructive setting, no infinite 
descending chains are involved, and hence the mathematically curious reader may wonder how 
the proof of this theorem goes. We present the key idea next -- the non-curious reader can 
skip directly to Section~\ref{ssect:new-induction}.

The proof of lemma \lstinline|wf_from_af| is done through the following generalization:
\inputcoq{AlmostFull}{AccFromAf}
To understand the generalization in lemma \lstinline|acc_from_af| it is easier to think again the 
special case of \lstinline|T| as a ``transition relation'' of a program. With lemma \lstinline|acc_from_af| 
we focus on a particular element \lstinline|y| and show that it is accessible. Instead of assuming that
the whole transitive closure of $T$ does not intersect with (the inverse of) \lstinline|R|, we will assume 
the same property only for the part of the transitive closure that ``follows'' \lstinline|y| in the sequence. 
Once this generalization is done, the proof is by induction on the \lstinline|almost_full| proof object.
\dv{I removed the informal text proof but the old proof is commented in text.}
%% We proceed by induction on \lstinline|p|: 
%% \begin{itemize*}
%%  \item 
%%   If \lstinline|p = ZT| then all elements are related by \lstinline|R|. It suffices to show that all elements
%%   \lstinline|z| for which \lstinline|T z y| are accessible (by the \lstinline|Acc_intro| constructor). But 
%%   then \lstinline|clos_refl_trans X T y y| and also \lstinline|clos_trans_1n X T z y| and \lstinline|R y z| 
%%   which is a contradiction, so this case is done.
%%  \item
%%   If \lstinline|p = SUP w| then, again, it suffices to show that all elements \lstinline|z| for which 
%%   \lstinline|T z y| are accessible (by the \lstinline|Acc_intro| constructor). We know that \lstinline|w y|
%%   secures the relation \begin{center} 
%%         \lstinline|Ry := fun y0 z0 => R y0 z0 \/ R y y0|
%%    \end{center}
%%   Thus, we may apply the induction 
%%   hypothesis for \lstinline|z|, instantiating \lstinline|R| to \lstinline|Ry|. To finish the case we need 
%%   to show that for any \lstinline|x| and \lstinline|w| such that \lstinline|clos_refl_trans X T w z|
%%   it is impossible to have \lstinline|clos_trans_1n X T x w /\ (R w x \/ R y w)|, given that it is impossible 
%%   to have \lstinline|clos_trans_1n X T x w /\ R w x| for all elements \lstinline|x| and \lstinline|w| with \lstinline|clos_refl_trans X T w y|. 
%%   Let us pick any 
%%   \lstinline|w| ``following'' \lstinline|z|. Since \lstinline|clos_refl_trans X T w z| and \lstinline|T z y| 
%%   it must be that \lstinline|clos_refl_trans X T w y|. So it cannot be that \lstinline|clos_trans_1n X T x w /\ R w x|. The 
%%   only case we have to rule out is when \lstinline|clos_trans_1n X T x w| and \lstinline|R y w|: But here 
%%   we have \lstinline|clos_refl_Trans X T y y| and \lstinline|clos_trans_1n X T w y| and \lstinline|R y w|, 
%%   which is again a contradiction.
%% \end{itemize*}

Deriving Lemma \lstinline|wf_from_af| from this generalization is a trivial task, since every element
is related to itself in the reflexive transitive closure of \lstinline|T|, hence every element is accessible, thus
\lstinline|T| is well-founded.

\subsection{A new induction principle}\label{ssect:new-induction}

If we can use lemma \lstinline|wf_from_af| to form WF relations, then we can surely use it to 
perform induction. The theorem \lstinline|af_induction| in Figure~\ref{t:compind} demonstrates 
a new induction principle, based on \lstinline|wf_from_af|. 
%\begin{center}
%\framebox{\inputcoq{AlmostFullInduction}{AfInduction}}
%\end{center}
Intuitively \lstinline|T| is the relation between the argument in the ``next'' recursive 
call (\lstinline|y|), and the previous (\lstinline|x|) and we are simply requiring that 
the transitive closure of \lstinline|T| has an empty intersection with (the inverse of) 
some AF relation \lstinline|R|. 

Hence, when using AF induction, the programmer must (i) provide an AF relation, (ii) show 
the emptyness of the intersection, and (iii) provide a functional for the recursive call. 
%\vspace{5pt}\begin{example}
%Let's see this induction principle in action with a very simple example that computes Fibonacci 
%numbers:\footnote{Of course \lstinline|af_induction| is a pretty contrived way to write this simple function 
%but we'd like to demonstrate the three steps with the simplest possible example before we dive in more complex examples.}
%\inputcoq{AlmostFullInduction}{Fibonacci}
%As a remark, to ensure that computation does not get stuck we have 
%used \lstinline|Defined| instead of \lstinline|Qed| (and we do so in most of our development), which makes the 
%various lemmas and definitions transparent for computation purposes.
%\end{example}
It is worth contrasting \lstinline|af_induction| with Coq's standard well-founded induction, 
and Figure~\ref{t:compind} presents our new induction principle side-to-side to Coq's standard WF 
induction.
\begin{figure*}[t]
\begin{tabular}{l}
\begin{lstlisting}
af_induction
     : forall (A : Set) (T R : A -> A -> Prop),
       almost_full R ->
       (forall x y : A, clos_trans_1n A T x y /\ R y x -> False) ->
       forall P : A -> Set,
       (forall x : A, (forall y : A, T y x -> P y) -> P x) -> forall a : A, P a
\end{lstlisting}\\ \\
\begin{lstlisting}
well_founded_induction
     : forall (A : Set) (T : A -> A -> Prop),
       well_founded T ->
       forall P : A -> Set,
       (forall x : A, (forall y : A, T y x -> P y) -> P x) -> forall a : A, P a
\end{lstlisting}
\end{tabular}
\caption{AF vs WF induction principles}\label{t:compind}
\end{figure*}
Notably, \lstinline|af_induction| takes both \lstinline|T| and an AF relation \lstinline|R|, the empty 
intersection requirement, and the functional for the recursion, whereas \lstinline|well_founded_induction| 
only requires a proof that \lstinline|T| is well-founded. 


\paragraph{Summary} So far we have generalized and proved the underlying principle behind WQO-based online
termination testing and revealed the connections between AF relations and WF relations. We have used this 
underlying principle to derive a simple AF-based induction principle. From a programmer's perspective this
new principle does not {\em yet} seem to have made things nicer really, as the user now has to prove two 
preconditions instead of one. So it's time we move on to the benefits of using AF relations that we 
promised to deliver in the introduction.

\section{Constructions on AF relations}\label{sect:af-constructions} 

The nicest characteristic of AF relations is their {\em remarkable composability}. In 
%this section we will show various results that substantiate this claim. by presenting a powerful toolkit 
%for the several constructions on AF 
%relations.
%the union of AF relations is AF (Section~\ref{ssect:union}) 
%and the intersection of AF relations is AF (Section~\ref{ssect:intersection}). 
%We also show type-based compositions (Section~\ref{ssect:typecomp}): how to use map-like operations, 
%how to create AF relations for products, tagged unions (sums), and finite types (such as \lstinline|bool|). 
Together with our results from Section~\ref{sect:wf-to-af}, which can be used to give us ``ground'' AF relations from existing WF relations, this section presents a powerful toolkit for the \lstinline|af_induction| user.

\subsection{AF unions}\label{ssect:union}

If we are presented with an infinite sequence in which there
exist two related elements by relation \lstinline|A| then obviously these two elements are also related 
by the relation \lstinline|A|~$\cup$~\lstinline|B|.More generally, it takes a straightforward induction on 
the AF proof objects to prove the following lemma:
\inputcoq{AlmostFull}{AFStrengthen}%
From which the next corollary follows:
\inputcoq{AFConstructions}{AfUnion}

\subsection{AF intersections}\label{ssect:intersection}

Intersections are much harder. Imagine that we have AF relations \lstinline|A| and 
\lstinline|B|. If we are presented with an infinite sequence then we definitely know that 
\lstinline|A| relates some elements in the sequence, and \lstinline|B| relates some elements in 
the sequence, but are there any elements that are {\em simultaneously} related by \lstinline|A| and 
\lstinline|B|? Remarkably, the answer is affirmative. A generalization of this theorem to $k$-ary AF relations is often called 
the ``intuitionistic version of Ramsey's theorem''~\cite{Veldman01041993}.\footnote{Exercise: use Ramsey's theorem 
to prove (classically) the intersection theorem, proceeding by contradiction and using a 3-coloring.} 

Here we focus on the binary case, following and simplifying the setup 
in~\cite{Veldman01041993,coquand-short}. We will not present the proof in detail in this section, 
but rather prove it as an instance of a much more general theorem in Section~\ref{sect:generalized-irt}.

%% The idea of the proof is that, given two well-founded 
%trees that secure relations \lstinline|A| and \lstinline|B| respectively, we will construct 
%another one that secures their intersection. This construction involves three stages. 
%
%\begin{itemize*} 
%  \item First, we define the \lstinline|oplus_nullary| function below:
%\inputcoq{AFConstructions}{OplusNullary}
%The function \lstinline|oplus_nullary| secures intersections of nullary predicates, which
%is shown with the following lemma:
%\inputcoq{AFConstructions}{OplusLemma}
%
%  \item Next, we proceed one level-up, to define \lstinline|oplus_unary|:
%\inputcoq{AFConstructions}{OplusUnary}
%We've written the function using Coq's tactic language because it involves a
%nested induction, but it's easier to understand it operationally using the 
%following Haskell code:
%\inpuths{AlmostFullHaskell}{OplusUnary}
%There is a similar lemma about \lstinline|oplus_unary|, that explains how it can be used
%to secure intersections of unary predicates: 
%\inputcoq{AFConstructions}{OplusUnaryLemma}
%
%  \item Finally, we proceed yet one level up, to define \lstinline|oplus_binary|: 
%\inputcoq{AFConstructions}{OplusBinary}
%Its definition follows \lstinline|oplus_unary| and it's perhaps simpler to understand in Haskell: 
%\inpuths{AlmostFullHaskell}{OplusBinary}
%\end{itemize*}
%
%The fixpoint \lstinline|oplus_binary| turns out to be exactly what we want to combine
%well-founded trees to secure intersections (of binary predicates). Here is the 
%corresponding lemma:
%\inputcoq{AFConstructions}{OplusBinaryLemma}
%The proofs of all three lemmas above are direct and short, and their corollary is:
\inputcoq{AFConstructions}{AfIntersection}

The binary version of the Ramsey theorem is, using classical logic, a direct consequence of \lstinline|af_intersection|: 
consider a binary relation $R$ on \lstinline|nat| and call a subset $A$ of \lstinline|nat| homogeneous iff:
\begin{itemize*} 
   \item For all $n$ and $m$ in $A$ such that $n < m$ it is $R\;n\;m$, {\em or} 
   \item For all $n$ and $m$ in $A$ such that $n < m$ it is $\lnot(R\;n\;m)$.
\end{itemize*}
Ramsey's theorem states that for every binary relation $R$ there exists 
an {\em infinite} homogeneous subset of \lstinline|nat|, $A$. To prove this, assume by 
contradiction that no such infinite homogeneous subset exists. This means that both $R$ and $\lnot R$ are AF, which 
means that their intersection is AF by \lstinline|af_intersection|. But the empty relation 
cannot be AF because it relates no elements whatsoever! 

As a final remark, the intersection theorem for the case of WQOs is folklore. There, 
the transitivity assumption appears to significantly simplify the 
proof. For instance, such a proof is contained in a short paper by Nash-Williams~\cite{OnWellQuasiOrderingFiniteTrees}.

\subsection{Type-based combinators}\label{ssect:typecomp}

In this section we show how to derive AF relations from simpler 
ones in a type-directed way, and how we may use them to define 
recursive functions. 

\paragraph{Ranking functions}
We can show a theorem that is useful when we would like to map 
complicated data structures to \lstinline|nat| values through ``ranking functions''.
\inputcoq{AFConstructions}{CoFmapLemma}
For instance we may map our data structures to natural numbers and re-use the 
$\leq$ relation and the \lstinline|leq_af| witness that $\leq$ is AF.

%We can show that well-founded trees is a cofunctor by defining
%a cofmap operation 
%\inputcoq{AFConstructions}{CoFmapLemma}
%with the straightforward property, and corollary: 
%\inputcoq{AFConstructions}{CofmapLemma}
%\inputcoq{AFConstructions}{CoFmapCorollary}
%For example, the \lstinline|af_cofmap| theorem can be used  when we would like to map 
%complicated data structures to \lstinline|nat| values through ``ranking functions'', so 
%that we may then re-use the $\leq$ relation and the \lstinline|leq_af| witness that $\leq$ is AF.

\vspace{5pt}
\begin{example}[Use of a ranking function]
Consider the following definition (in Haskell notation\footnote{AFExamples.v 
gives the Coq definition.}):
\begin{lstlisting}[language=hs]
flip1 (0,_) = 1 
flip1 (_,0) = 1 
flip1 (x+1,y+1) = flip1 (y+1,x)
\end{lstlisting}
Through the use of \lstinline|af_cofmap| we may define this function by observing that
the transition relation is \lstinline|T x y := fst x <= snd y /\ snd x < fst y|. We may now
take \lstinline|R x y := fst x + snd x <= fst y + snd y | as our AF relation. Showing that 
\lstinline|forall x y, clos_trans T x y /\ R y x -> False|
is easy and the proof that \lstinline|R| is AF is just (\lstinline|af_cofmap leq_af|).
\end{example}

\paragraph{Finite types}

There is a very natural AF relation on types that have finitely many inhabitants, and that 
is simply the equality on elements of these types. The simplest interesting such finite type
is \lstinline|bool|. Why is equality on booleans AF? Because in any infinite sequence we are 
guaranteed that in the first three elements of the sequence two of them will be equal. Hence
we can show:
\inputcoq{AFConstructions}{AfBool}
and the proof involves three applications of \lstinline|AF_SUP| followed by a \lstinline|AF_ZT|. 

We are not going to generalize here this construction to arbitrary finite types, but the
reader should be convinced that this is possible to do -- a proof object with $k+1$ uses of 
\lstinline|AF_SUP| before returning \lstinline|AF_ZT| does the job for any finite type 
inhabited by $k$ values. Our accompanying development includes this construction. 

\paragraph{Products}
The intersection property and cofunctoriality are already extremely powerful -- here is 
the simplest construction to create an AF relation for products based on these components:
\inputcoq{AFConstructions}{AfProduct}
The proof is just applications of \lstinline|af_intersection| and \lstinline|af_cofmap| through
the \lstinline|fst| and \lstinline|snd| projections out of pairs.

Of course, this is not the only AF relation on products -- it's just a particular one. For
instance one could completely ignore the second component of a pair and only use and AF 
relation on the first component though the use of \lstinline|af_cofmap|.

%\inputcoq{AFConstructions}{AfProductLeft} 
%or could imagine other wild combinations through the use of cofunctoriality, unions and 
%intersections -- or as a last resort, could create something more sophisticated by hand-writing 
%the AF proof for our relation.

We give now a small example to demonstrate the product combinator in action. 
\vspace{5pt}\begin{example}[Lexicographic order]
Consider the following recursive definition (in Haskell notation):
\begin{lstlisting}[language=hs]
flex (0,_) = 1
flex (_,0) = 1 
flex (x+1,y+1) = f (x,y+2) + f (x+1,y) 
\end{lstlisting}
This is an example of function where the arguments descend lexicographically. We can 
also observe that in any recursive call, one of the two arguments is decreasing. This 
immediately suggests that we should use the AF 
relation 
\begin{center} 
\lstinline| R x y := fst x <= fst y  /\ snd x <= snd y|
\end{center}
Recall that since $\leq$ is AF and we have already given a product combinator (\lstinline|af_product|), 
the relation \lstinline|R| is AF. The transition relation of the program is also what you'd expect:
\lstinline|T x y := fst x < fst y \/ (fst x = fst y /\ snd x < snd y)|, since in the first recursive call, 
the first argument becomes smaller, and in the second recursive call the second argument becomes smaller, while the first remains the same.
It is then quite easy to show that
\lstinline|forall x y, clos_trans T x y /\ R y x -> False| and 
apply \lstinline|af_induction| to define the recursive function.
\end{example}

\paragraph{Sums} 
If we are given two AF relations on types \lstinline|X| and \lstinline|Y| respectively, 
is there a natural AF relation that we can define on \lstinline|X+Y|? One that we find 
often useful is the relation that lifts these two relations in the following way:
\inputcoq{AFConstructions}{SumLift}
If two elements have the same tags they are compared with one or the other relation, 
otherwise they are not related. We will now show that if \lstinline|A| and \lstinline|B|
are AF, then so is \lstinline|sum_lift A B|. The key intuition behind our construction is
the close connection between tagged sums and products where the first component is the 
``tag'' and the second is the value, and is ommited due to lack of space. Our result is:
\inputcoq{AFConstructions}{AfSumLift}

%Here is the construction step-by step.
%\begin{itemize*}
%  \item First, consider the following definition
%  \inputcoq{AFConstructions}{LeftSumLift}
%  which is almost like \lstinline|sum_lift| but in the case of two right injections returns
%  \lstinline|True|. 
%  If we are given a proof object that shows the relation \lstinline|A : X -> X -> Prop| is AF, 
%  we can show that \lstinline|left_sum_lift A| is AF. 
%  \inputcoq{AFConstructions}{SecLeftSumTree}
%  What is the intuition behind the proof of \lstinline|af_left_sum|? If the proof object is \lstinline|AF_ZT| then all 
%  elements \lstinline|X| are related and hence if we take three elements in a row with 
%  \lstinline|AF_SUP| constructors we are guaranteed to have shown \lstinline|left_sum_lift A| to be AF, 
%  either  by having met two left injections, or by having met two right injections. This is exactly what
%  we did in the case of finite types. 
%  If on the other hand the proof object that shows \lstinline|A| to be AF is \lstinline|AF_SUP f|, then we 
%  examine the next element \lstinline|x| -- if it is a left injection we may simply recurse.
%  If \lstinline|x| is a right injection, then we examine the subsequent element \lstinline|y|: 
%  if \lstinline|y| is a left injection we can again recurse, but if it is also a right injection
%  then we are simply finished (since all right injections are related by \lstinline|left_sum_lift A|).
%
%  \item Our next step is to flip everything around! We will use of a \lstinline|transpose| 
%  function and get symmetric versions of the previous lemma.
%  \inputcoq{AFConstructions}{Transpose}
%  \inputcoq{AFConstructions}{RightTranspose}
%  \inputcoq{AFConstructions}{SecRightSumTree}
%  
%  \item We are almost there! Observe that if we are given relations
%        \lstinline|A : X -> X -> Prop| and \lstinline|B : Y -> Y -> Prop| then the 
%        {\em intersection} of the \lstinline|left_sum_lift A| and \lstinline|right_sum_lift B| 
%        is precisely \lstinline|sum_lift A B|. We've already shown a combinator for intersections
%        of relations so it's now straightforward to derive our final result
%        \inputcoq{AFConstructions}{AfSumLift}
%\end{itemize*}

We do not present here an example of using \lstinline|af_sum_lift|, but we will see an example
later in Section~\ref{sect:mutual}. \dv{will we or we don't have space?} 

\paragraph{Dependent products and recursive types}
We do no currently include in our development combinators for dependent products
nor recursive types, though nothing seems to be prohibitive about either and we 
intend to extend the set of available type-based combinators in the future. The 
question of recursive types is of particular interest as it is a well-studied topic
in the context of WQOs, where the canonical WQO for lists and more general recursive 
types is based on homeomorphic embeddings~\cite{OnWellQuasiOrderingFiniteTrees,TreeTheorem}. 
There have been attempts to port some of these theorems for WQOs in a constructive setting 
(e.g. homeomorphic embeddings for lists on finite alphabets~\cite{berghofer-higman,Seisenberger}) 
so we believe this is a quite plausible direction for future work. 

\section{Size-change termination and AF induction}\label{sect:af-in-practice}

We have examined combinators on AF relations, and simple examples such as lexicographic
descent. Lexicographic orders are not terribly difficult (Coq already comes with 
combinators to compose lexicographically two well-founded relations, in fact) but the power of 
the method shows itself in examples that go beyond lexicographic orders. Consider 
the following example.

\vspace{5pt}\begin{example}[Beyond lexicographic order]
Here is an example in Haskell-like notation:
\begin{lstlisting}[language=hs]
gnlex (0,_) = 1 
gnlex (_,0) = 1
gnlex (x+1,y+1) = gnlex (y+1,y) + gnlex (y+1,x)
\end{lstlisting}
To define this program, we will use the AF \lstinline|R| for products, and the 
``obvious'' transition relation \lstinline|T|:
\inputcoq{AFExamples}{GNLexRelations}
It's now possible to show that the transitive closure of \lstinline|T| has 
an empty intersection with (the inverse of) \lstinline|R|. Thus, our accompanying development 
defines \lstinline|gnlex| using \lstinline|af_induction|. 
\end{example}

%we have proved this 
%lemma independently and called it \lstinline|T_empty_intersect|. Using this 
%lemma we may define \lstinline|gnlex| completely as: 
%\inputcoq{AFExamples}{GNLexSimple}
%\end{example}

Ben-Amram~\cite{Ben-amram02generalsize-change} notices that examples like \lstinline|gnlex| belong 
in a syntactic class of programs that can be shown terminating by {\em size-change termination} (SCT) ~\cite{Lee+:sct, jones-bohr,Ben-amram02generalsize-change}  
but not by a direct lexicographic descent argument, although semantically the class of {\em mathematical 
functions} one may define using size-change termination and those that can be defined with 
lexicographic descent orders coincide. It is then reassuring to see that examples from that 
class can be written quite straightforwardly!

\subsection{Formal connection}

In fact, the connection to size-change termination can be made more precise. The short 
summary of this section is that the soundness of size-change termination follows from our 
general \lstinline|wf_from_af| lemma. For the rest of this section we show this connection, 
using \lstinline|gnlex| as our working example. 

Let us start by briefly describing the main idea behind SCT: The first step 
in showing that a recursive definition is terminating is to identify the various 
recursion patterns and abstract each as a {\em size-change graph}. A size-change graph for a $k$-argument function is a 
labeled graph with nodes labeled from $\{0,\ldots,k-1\}$ and arcs with labels $<$ and $\leq$.
\vspace{5pt}
\begin{example}[Size-change graph for gnlex]
For our two-argument function \lstinline|gnlex| we get the following 
two size-change graphs (arising from the transition relation \lstinline|T| of the function):
\newcommand{\tZ}{{\footnotesize{0}}}
\newcommand{\tO}{{\footnotesize{1}}}
\tikzstyle{nsct}= [circle, draw, minimum size = 5pt,inner sep=1pt, outer sep=1pt]
\begin{center}
\begin{tabular}{ccc}
\begin{tikzpicture}[auto, sibling distance = 30mm, node distance=1.5cm]
    \node [nsct] (a) {\tZ};
    \node [nsct, left of=a] (b) {\tZ};
    \node [nsct, below of=a] (c) {\tO}; 
    \node [nsct, below of=b] (d) {\tO};
    \path [->] (b) edge node[sloped] {$\leq$} (c);
    \path [->] (d) edge node[sloped] {$<$} (c);
\end{tikzpicture} & \quad & 
\begin{tikzpicture}[auto, sibling distance = 30mm, node distance=1.5cm]
    \node [nsct] (a) {\tZ};
    \node [nsct, left of=a] (b) {\tZ};
    \node [nsct, below of=a] (c) {\tO}; 
    \node [nsct, below of=b] (d) {\tO};
    \path [->,auto] (b) edge node[sloped] {$\leq$} (c);
    \path [->,auto] (d) edge node[sloped] {$<$} (a);
\end{tikzpicture} \\ 
$G_0$ & \quad & $G_1$ 
\end{tabular}
\end{center}
\end{example}
A size-change graph $G$ for a $k$-argument function induces a relation on $k$-tuples 
and we say that a size-change graph {\em approximates} a relation \lstinline|T| 
iff $T \subseteq T_{G}$. In our \lstinline|gnlex| example, each of the two graphs approximates 
a disjunct from \lstinline|T|. 

Size change graphs {\em compose} so that the composition of two arcs one of which is $<$ 
creates a new arc $<$, whereas the composition of two $\leq$ arcs gives a new $\leq$ arc. 
We write this composition with notation $G_1;G_2$ (written $G_{12}$ for brevity
below). Graph composition satisfies the following proposition.

\vspace{5pt}\begin{proposition}
If $G_1$ approximates $T_1$ and $G_2$ approximates $T_2$ then $G_1;G_2$ approximates $T_1{\cdot}T_2$
(where $\cdot$ is transitive relation composition).
\end{proposition}

Assume now that the transition relation of a program is given by $n$-disjuncts 
$T = T_1\cup\ldots T_n$ each of which corresponds to some recursion pattern and is 
approximated by a size-change graph $G_i$ (as in our example with $n = 2$). Size-change 
termination then considers the set $S$, defined as the transitive closure
of the set $\{ G_1,\ldots,G_n \}$ under graph composition.
\vspace{5pt}
\begin{example}[Transitive closure of size-change graphs]
What is this set $S$ in our \lstinline|gnlex| example? If we start off with 
$G_0$ and $G_1$, we have to consider the compositions $G_0;G_0$, $G_0;G_1$, $G_1;G_0$, 
and $G_1;G_1$. We observe that 
$G_{00}$ is a new graph with edges $ 0 \stackrel{<}{\longrightarrow} 1$, $1 \stackrel{<}{\longrightarrow} 1$, 
$G_{01}$ is a new graph with edges $ 0 \stackrel{<}{\longrightarrow} 0$, $1 \stackrel{<}{\longrightarrow} 0$, 
$G_{10}$ is {\em exactly} $G_{00}$ and $G_{11}$ is a new graph with 
edges $ 0 \stackrel{<}{\longrightarrow} 0$, $1 \stackrel{<}{\longrightarrow} 1$. If we continue in this fashion
we can compute that the set $S$ is just:
\[ S = \{ G_0, G_1, G_{00}, G_{01}, G_{11}, G_{111} \} \] 
\end{example}

What is the importance of the set $S$? We have seen that $T$ can be approximated
by $\{G_0,G_1\}$ and we have seen that compositions of graphs approximate compositions of 
relations. This means that for every $k$, the composition of $T$ with itself $k$ times
$T^k$ (which we will call the $k$-th {\em power} of $T$) can be approximated by the set 
of graphs in $S$ (which will typically, as in our example, be finite): Precisely, for 
every $x$ and $y$ such that $T^{k}\;x\;y$ it is the case that $T_{G}\;x\;y$ for 
some $G \in S$. This is wonderful news, because it enables the following lemma. 

\vspace{5pt}\begin{lemma}\label{lem:general-sct}
Assume that $T = T_1\cup\ldots\cup T_n$ and $G_i$ approximates $T_i$, and let $S$ be the transitive 
closure of the set $\{G_i,\ldots,G_n\}$. If every $G \in S$ induces a relation $T_G$ such that
$T_G \cap R^{-1} = \emptyset $ for some AF $R$ then $T$ is well-founded.
\end{lemma} 
\begin{proof} 
By \lstinline|wf_from_af| we only have to show that for all $x$ and $y$ 
such that $T^{+} x\;y$ it is not the case that $R\;y\;x$. If
$T^{+}x\;y$ then there exists some $k$ such that $T^{k}x\;y$, which means that 
there exists some $G \in S$ such that $T_G\;x\;y$ and we know that $T_G \cap R^{-1} = \emptyset$.
\end{proof}

Next, consider the size-change graph $I$ with 
edges $i \stackrel{\leq}{\longrightarrow} i$ for each $i$, and let us call the induced relation 
$T_I\;x\;y = \bigwedge x_i \leq y_i$. By the constructive Ramsey Theorem, \lstinline|af_intersection|, $T_I$ is AF.

\vspace{5pt}\begin{example}
We can now show that \lstinline|gnlex| is terminating by checking that 
every graph $G \in S$ has empty intersection with $(T_I)^{-1}$ and using 
Lemma~\ref{lem:general-sct}.
\end{example}

Size-change termination uses the same AF relation $T_I$ and Lemma~\ref{lem:general-sct}, 
through the following auxiliary lemma. 

\vspace{5pt}\begin{lemma}\label{lem:aux-sct}
If $G$ approximates $T$ and some power $G^n$ of $G$ contains an arc $i \stackrel{<}{\longrightarrow} i$ 
then $T \cap T_I^{-1} = \emptyset$. 
\end{lemma}
\begin{proof} Assume that $T\;x\;y$ and $T_I\;y\;x$. We then have 
$(T{\cdot}T_I)\;x\;x$ and $(T{\cdot}T_I)^n\;x\;x$. But $I$ approximates $T_I$ and because compositions
of graphs approximate compositions of relations and $G;I = G$ it follows 
that $G^n$ approximates $(T{\cdot}T_I)^n$. this means that $x_i < x_i$, which is a 
contradiction.\footnote{The argument is still constructive as we can represent 
$T \cap T_I^{-1} = \emptyset$ as $\forall x y, T~x~y \land T_I~y~x \to {\tt False}$.}
\end{proof}
We are now ready to state and prove the basic SCT principle. 
\vspace{5pt}\begin{theorem}[Size-change termination]
Assume that $T = T_1\cup\ldots\cup T_n$ and $G_i$ approximates $T_i$, and let $S$ be the transitive 
closure of the set $\{G_i,\ldots,G_n\}$. If every $G \in S$ has a power with an 
arc $i \stackrel{<}{\longrightarrow} i$ then $T$ is well-founded. 
\end{theorem}
\begin{proof}
By Lemma~\ref{lem:aux-sct} we know that $T_G \cap T_I^{-1} = \emptyset$ for every $G \in S$, and 
by Lemma~\ref{lem:general-sct} we are done. 
\end{proof}
Hence, we have proved the size-change termination condition relying on our 
\lstinline|wf_from_af| theorem. The reader can observe that the condition is true 
for the set $S$ we have computed for \lstinline|gnlex|. 

As a side-note, sometimes the size-change termination criterion is stated by requiring that 
every idempotent graph $G \in S$ has an arc $i \stackrel{<}{\longrightarrow} i$, which is an equivalent 
condition since any size-change graph has an idempotent power. 

We have not formalized this connection in Coq, but this formalization seems like an interesting
direction for future work, especially combined with tactics to extract automatically 
size-change graphs from Coq recursive definitions. For now we will simply state that SCT 
can be proved using our \lstinline|wf_from_af| and leave the development of an SCT-based 
tactic as future work. Finally, the literature on SCT is also concerned 
with {\em mutual induction}. We show how to define mutually inductive fixpoints using AF 
relations in Section~\ref{sect:mutual}.

%% this amounts to checking that all idempotent compositions of size-change graphs have
%% an empty intersection with the (inverse of) the AF relation which is comprised of the intersection 
%% of simpler AF relations for each of the arguments. It seems that this principle is more powerful 
%% than the \lstinline|af_power_induction|, but at the same time it seems to be better suited for a 
%% tool and not for humans: To use SCT, one has to identify all possible loops in compositions of 
%% size-change graphs. That said, we've shown how to use \lstinline|af_power_induction| to show a
%% canonical example from size-change termination and we leave it as future work to formulate SCT 
%% termination as an extension of our library. 

\section{The Terminator rule}\label{sect:terminator}

We have used online termination and WQOs as a way to approach AF relations and 
\lstinline|af_induction|, but it turns out that \lstinline|af_induction| is general 
enough to capture the proof principle behind Terminator~\cite{terminator,podelski-rybalchenko:transition}.
The key theorem behind Terminator is the {\em disjunctive well-foundedness} statement below:
\begin{quote} If $R_1\ldots R_n$ are well-founded for some finite $n$, 
              and $R^{+} \subseteq R_1 \cup \ldots \cup R_n$ then $R$ is well-founded.
\end{quote}
The proof of this theorem relies on a Ramsey argument~\cite{podelski-rybalchenko:transition}, 
but here we will simply prove it -- intuitionistically -- by using theorem \lstinline|wf_from_af| 
from Section~\ref{sect:af-to-wf}. Here it is, together with the proof: 
\inputcoq{Terminator}{DisjunctiveWF}
For the sake of demonstration, we have stated and proved the theorem 
when $n = 2$ but a generalization is trivial. The proof is instructive as well: we do it
by appealing to \lstinline|wf_from_af| (Section~\ref{sect:af-to-wf}), and instantiating: 
\begin{center}
\lstinline| R := fun x y => not (R1 y x) /\ not (R2 y x)| 
\end{center} 
We next show that \lstinline|R| is AF by using our constructive Ramsey \lstinline|af_intersection| 
lemma and \lstinline|af_from_wf| (Section~\ref{sect:wf-to-af}). Finally the intersection emptyness 
condition is trivial to show. 

We can easily then use disjunctive well-foundedness to deduce the standard Terminator proof rule (for 
the union of two WF relations): 
\begin{center}
\framebox{\inputcoq{Terminator}{TerminatorInduction}}
\end{center}

\section{Generalized Ramsey's theorem}\label{sect:generalized-irt}

We now turn our attention to the intersection theorem for AF relations, given in 
Section~\ref{ssect:intersection}, which consitututes an intuitionistic version of Ramsey's theorem. We will show here
a much more general result for relations of ``inductive arities''. We start with defining predicates on lists
\inputcoq{AlmostFullGeneralized}{LRel}
We now generalize our \lstinline|almost_full| definition for predicates over lists:
\inputcoq{AlmostFullGeneralized}{AFLRel}
Again there exist two constructors -- the \lstinline|AF_ZT| constructor asserts that the predicate is true for every 
list, whereas the \lstinline|AF_SUP| constructor asserts that when presented with one new element \lstinline|x|, the
predicate \lstinline|fun ys => R ys \/ R (x :: ys)| is AF.
We will be interested in predicates over lists which have an ``inductive arity'': informally, for every sequence of input we
may append new values only a finite number of times before the value of the predicate becomes constant. We formalize
this with the definitions below:
\inputcoq{AlmostFullGeneralized}{Arity}
The inductive datatype \lstinline|WFT| encodes well-founded trees over a set \lstinline|X|. The \lstinline|Arity| fixpoint
defines when a relation \lstinline|A| has an ``inductive'' arity.  If we are given a \lstinline|ZT| well-founded tree then the 
truth value of the predicate \lstinline|A| is 
constant. However if we are given a \lstinline|SUP w| tree then for every element \lstinline|x|, the relation 
\lstinline|fun ys => A (x::ys)| has an inductive arity. One can see that since \lstinline|WFT| is inductive this fixpoint ensures
that after a finite number of inputs (which nevertheless depends on the individual input sequence each time) the value of
the predicate will become constant. 

Our main result is the following:
\inputcoq{AlmostFullGeneralized}{AFIntersectLRelCor}
It's proof is done by generalizing the statement to:
\inputcoq{AlmostFullGeneralized}{AFIntersectLRel} 
and proceeding by induction on the arity witness \lstinline|p|, and then the AF proof for \lstinline|A|, and then the 
AF proof for \lstinline|B|. 

How can we move from and to these generalized list predicates with inductive arities? For instance, for binary relations
we may define: 
\inputcoq{AlmostFullGeneralized}{BinRelExtension}
In this example we extend a binary relation to an \lstinline|LRel| by asserting that the value of the predicate on any list
with length greater than 2 is the value of the relation on the first 2 arguments. The value of the predicate on fewer arguments
is just \lstinline|False|. With this definition in place we have proved the following results:
\inputcoq{AlmostFullGeneralized}{Af2AfLrel}
\inputcoq{AlmostFullGeneralized}{AfLrel2Af}
from which the usual \lstinline|af_intersection| theorem from Section~\ref{ssect:intersection} follows.

\section{Mutual induction variations}\label{sect:mutual}

%Our accompanying development includes other variations of induction principles which often rely
%on finite types or our combinators for sums.  

%Recall that, given a transition relation \lstinline|T| and an AF relation \lstinline|R|, 
%our proof obligation for \lstinline|af_induction| is:
%\begin{center}
%  \lstinline|forall x y, clos_trans_1n X T x y /\ R y x -> False|
%\end{center}
%Often, it might be tedious to find the right generalization of this 
%statement to prove it inductively on the transitive closure. For this reason 
%we include a powerful generalization of our induction principle, which we 
%call {\em power-induction}: 
%\begin{center}
%\framebox{\inputcoq{AlmostFullInduction}{AfPowerInduction}}
%\end{center}
%We use notation \lstinline|power k T| for the $k$-th transitive composition 
%of \lstinline|T| with itself (written earlier as $T^k$). For instance, 
%\begin{center}
%\begin{tabular}{ll}
%    \lstinline|power 1 T = fun x y => T x y| \\ 
%    \lstinline|power 2 T = fun x y => exists z, T x z /\ T z y| 
%\end{tabular} 
%\end{center}
%and so on. 
%The power of \lstinline|af_power_induction| is that it allows for transition relations
%that don't immediately exhibit some argument metric going down, but they do so after 
%some ($k$) recursive calls. This is akin to {\em inlining} a recursive definition.
%
%Here is another example that can be programmed nicely using power-induction and the 
%sum combinators we've presented earlier. 

%\vspace{5pt}\begin{example}[Induction and sum combinators]
%Our development includes the following example (presented here in Haskell-like notation): 
%\begin{lstlisting}[language=hs]
%fsum (inl 0)         = 1 
%fsum (inl (S x))     = fsum (inr (x+2)) 
%fsum (inr x) | x < 2 = 0 
%fsum (inr x)         = fsum (inl (x-2))
%\end{lstlisting}
%%The interesting observation in this example is that, starting from a left injection, 
%%after two recursive calls we are back at a left injection and the value has 
%%decreased. Similarly for a right injection. Let us use this intuition and define
%%the AF relation \lstinline|Rfsum|, which uses our sum combinator \lstinline|sum_lift|, 
%%along with the (obvious) transition relation \lstinline|Tfsum| below: 
%%\inputcoq{AFExamples}{FSumRelations}
%%Notice that \lstinline|(power 2 T)| satisties a nicer invariant (under transitive composition) 
%%than \lstinline|T| and we may use AF induction for the \lstinline|(power 2 T)| relation: 
%%\inputcoq{AFExamples}{FSum}
%%Notice that we use $k=2$ and we rely on an independently proven 
%%lemma about the intersection emptyness, \lstinline|fsum_pow2_empty|.
%%Actually, we can repeat the same definition by using ordinary AF 
%%induction ($k = 1$) but doing so requires us to prove a more complicated 
%%invariant about \lstinline|Tfsum|. Happily, choosing $k = 2$ makes the 
%%proof of the intersection emptyness much simpler.
%\end{example}

We can easily derive mutual induction principles using AF induction, and we outline here 
the basic idea for two mutually recursive functions -- the accompanying development gives 
the full details. 

Suppose that we want to define two mutually recursive functions \lstinline|f : A -> C| 
and \lstinline|g : B -> D|. Suppose that the transition relation for calls from \lstinline|f| 
to itself is \lstinline|TAA:A->A->Prop| and for calls from \lstinline|f| to \lstinline|g| is 
\lstinline|TBA:B->A->Prop|. Similarly calls from \lstinline|g| to itself are described
by \lstinline|TBB:B->B->Prop| and calls from \lstinline|g| to \lstinline|f| are 
described by \lstinline|TAB:A->B->Prop|. We may consider the sum \lstinline|A + B|
and ``lift'' the relations \lstinline|TAA|, \lstinline|TBA|, \lstinline|TAB|, \lstinline|TBB| with 
\lstinline|lift_rel_union TAA TBB TAB TBA| which operates on \lstinline|A + B| in the following way: if 
both arguments  are left injections use \lstinline|TAA|, if they are both right 
injections use \lstinline|TBB|; otherwise use \lstinline|TAB| and \lstinline|TBA| for each case.

If we are given two almost-full relations \lstinline|RA:A->A->Prop| and \lstinline|RB:B->B->Prop| then 
we can show that the mutually recursive definition is well-formed when
\begin{lstlisting}
forall x y, clos_trans_1n (A+B) 
               (lift_rel_union TAA TBB TAB TBA) x y /\ 
               sum_lift RA RB y x -> False
\end{lstlisting}
Recall that \lstinline|sum_lift| returns \lstinline|False| if the tags do not match, otherwise compares 
the arguments using \lstinline|RA| or \lstinline|RB|. Intuitively, we tag each argument with the ``name'' 
of the function it is passed to, and we require that, whenever we return to an argument with the same 
tag (transitively) the intersection with an AF relation be empty. 

Our development derives such a mutual induction principle (but slightly more general), called 
\lstinline|af_mut_induction| and shows how one can use it to define fixpoints like:
\begin{lstlisting}[language=hs]
f 0     = 1 
f (x+1) = f x + g (x+2)
g x | x < 2 = 1
g (x+2)     = f x 
\end{lstlisting}
The interesting bit in this example is that the argument $x+1$ in the first recursive
call to $g$ does not decrease -- in fact it increases and only when we return to a call
to $f$ through $g$ does it decrease. 

The design of convenient and general mutual induction principles is a topic that is subject 
to many engineering decisions and seems an excellent direction for future work. For instance, 
we could define very easily a $k$-ary mutual induction principle on $k$ functions that all 
accept arguments of type \lstinline|A| by using a similar methodology: lift the transition relations 
to type \lstinline|(A * Finite k)| using the finite types to tag arguments with function identifiers, and 
lift an AF relation on \lstinline|A| to \lstinline|(A * Finite k)| using the \lstinline|af_intersection| 
theorem and the fact that equality on finite types is an AF relation. 

%\paragraph{The computational content of the proofs}
%
%Our development is constructive, so one might wonder about the 
%computational content of our proofs. Recall corollary \lstinline|af_inf_chain| from 
%Section~\ref{ssect:af}:
%\inputcoq{AlmostFull}{InfiniteChainCorollary}
%In fact we can write a function that, given a \lstinline|WFT X| tree \lstinline|p|, 
%and an infinite sequence \lstinline|f : nat -> X| computes the length of a prefix of
%the sequence in which there exist two related elements: 
%\begin{lstlisting}
%Fixpoint aux_length X (p:WFT X) (f:nat->X) (k:nat) := 
%  match p with 
%  | ZT => k 
%  | SUP g => aux_length (g (f k) f (S k)).
%  end.
%Definition length X (p : WFT X) f := aux_length p f O 
%\end{lstlisting}
%In \lstinline|aux_length|, the variable \lstinline|k| is the cursor into 
%the sequence \lstinline|f|, as in the proof of \lstinline|sec_binary_infinite_chain| 
%in Section~\ref{ssect:af}.
%
%One expects the \lstinline|length| function to terminate immediately when \lstinline|f| 
%is the trivial sequence $0,0,\ldots$ (\lstinline|fun (x:nat) => O|) and \lstinline|p| is 
%the well-founded tree that secures $\leq$ (let us call this \lstinline|leq_wft|). Indeed that 
%is the case, since our construction of the well-founded tree for $\leq$ is based on comparing 
%two consecutive elements for $<$.
%
%Surprisingly, if \lstinline|p = oplus_binary leq_wft leq_wft| (which secures the same 
%relation, $\leq$) then the call to \lstinline|length p (fun x => O)| does not terminate almost
%immediately, does not slow down as computation continues, and does not consume increasingly 
%more memory (in Coq or Haskell, where we've also implemented this for comparison): it seems
%to loop! 
%
%Of course it doesn't {\em actually} loop -- the reason for this behavior is the exponential nature 
%of the combinators to form trees for intersections of relations. Notice that 
%\lstinline|length p (fun x => O)| gives the length of the left-most path in the 
%well-founded tree \lstinline|p| and we can compute the size of this left-most path following the structure 
%of \lstinline|oplus_nullary|, \lstinline|oplus_unary|, and \lstinline|oplus_binary|:
%{\small 
%\begin{lstlisting}[language=hs]
% lm_nullary x y = x + y 
% 
% lm_unary 0 y = y 
% lm_unary x 0 = x 
% lm_unary (x+1) (y+1) = 1+lm_nullary (lm_unary x (y+1)) 
%                                     (lm_unary (x+1) y) 
% lm_binary 0 y = y 
% lm_binary y 0 = y 
% lm_binary (x+1) (y+1) = 1+lm_unary (lm_binary x (y+1)) 
%                                    (lm_binary (x+1) y)
%\end{lstlisting}
%}
%
%The leftmost path of \lstinline|leq_wft| is just 2 and hence we are interested
%in \lstinline|lm_binary 2 2|, which gives us the enormous bound 1254125905363099368618480!
%
%This discussion suggests that our combinators for composing AF relations cannot be directly used 
%to replace or improve history-based online termination testing. For instance, the resulting 
%termination test for the sequence $(0,0),\ldots$ (which is based on intersection of relations $\leq$ for 
%the first and second components of a pair) would consume very little memory but would reject the 
%sequence after 1254125905363099368618480 elements in the sequence!
%
%It seems an interesting direction to explore whether there exists a more ``efficient'' version 
%of these combinators. On the other hand, for type theory (or Coq) these enormous bounds do not 
%appear to cause any problems at all. 

%\paragraph{Prop versus Set witnesses}
%
%The previous discussion about the computational content of our proofs raises another question. 
%Why did we separate well-founded trees from the \lstinline|SecureBy| predicate, and didn't we 
%simply define a single inductive predicate for both: 
%\begin{lstlisting}
%Inductive AF X : (X -> X -> Prop) : Prop := 
% | AF_ZT :forall R, (forall x y, R x y) -> AF R 
% | AF_SUP : 
%     forall R, (forall x, AF (fun y z => R y z \/ R x y)) -> AF R.
%\end{lstlisting}
%There is no significant obstacle associated with this definition but 
%we have not carried out the experiment -- following a similar stratification in previous work.
%Actually, it may be advantageous for code extraction to have arguments that only live in \lstinline|Prop|. 
%On the other hand, having to deal with concrete \lstinline|Set|-based witnesses (\lstinline|WFT X|) felt
%reassuring and made porting some of this development to Haskell quite straightforward when 
%investigating the computational content of the AF combinators. 

%% \paragraph{Optimizations using transitivity}
%% It is actually not necessary to check the whole transitive closure of a relation
%% to check if the intersection of this and the almost-full relation are empty. It 
%% suffices, for a transitive almost-full relation (a well-quasi-order), to check a 
%% subset of it, see {\tt FilteredTransitiveClosure.v} for the details.


\section{Related work}\label{sect:related}

\dv{Must revise and add pointers from Carsten Fuhs. I've already added the citations on the local.bib file so we 
just need to read and say a few words about these works.}

We have already seen the proof principle 
behind Terminator~\cite{terminator,podelski-rybalchenko:transition} in 
Section~\ref{sect:terminator}. But why has that proof principle been so successful? One 
answer is {\em composability}: The way the tool works in practice is by rewriting the 
program to capture the transitive closure of the transition relation $R$ using some new 
program variables
and then try to synthesize well-founded relations $R_1\ldots R_n$ so that $R^{+} \subseteq R_1\cup\ldots\cup R_n$ from
static analysis of the code. The way this synthesis works is by starting off with the empty 
union, for which $R^{+} \not\subseteq \emptyset$, a fact that can be used to derive a well-founded
relation $R_1$. Next, a similar check is made that $R^{+} \subseteq R_1$. If this test fails,
Terminator uses the failed proof to discover yet another well-founded relation $R_2$ and 
this time check $R^{+} \subseteq R_1\cup R_2$. The process continues until a termination argument is
found. The key to the success of this method has been that the termination test simply 
uses unions of well-founded relations (instead of trying to discover more complex ways to 
compose them) and throws the ``hard'' part of the proof of checking that ($R^{+} \subseteq R_1\cup\ldots\cup R_n$) 
to extremely powerful external tools such as SMT solvers. Interestingly, Terminator
performs something like AF-power-induction if the termination arguments fail
for the transition relation: it starts unrolling loops until termination argument synthesis
is able to find an answer. 

Porting some of Ramsey theory in a constructive setting seems to have been a fascinating 
subject among mathematicians and computer scientists, since the original proof and 
definitions seem hopelessly classical. Our development is based on Veldman's original 
ideas~\cite{Veldman01041993,coquand-short}. This is not to say that our development 
is the only possible way to develop constructive 
Ramsey-like arguments, for instance there exists an alternative formulation~\cite{Coquand:1994:ART:185268.185270} but does 
not seem as suitable for termination purposes as the one we present in this paper. Similarly, constructive proofs of 
various homeomorphic embedding lemmas (such as Higman's Lemma~\cite{berghofer-higman,Seisenberger,Fridlender97higmanslemma})
have appeared in the literature. Our development seems to be the first that connects 
constructive Ramsey theory and termination proving. 

Nowadays there exists a large set of recursion-encoding techniques in 
type theory and Coq, some of which include good support for automation.
The most straightforward way to program recursion in Coq~\cite{coqart} is either by 
structural recursion or by using subset types~\cite{Sozeau06subsetcoercions} 
and \lstinline|measure| arguments. An extension of ``guarded'' recursion (and co-recursion)
implemented in a variant of Agda is sized-types~\cite{abel:tlca03} (not to be confused with 
size-change termination).

The Bove and Capretta method~\cite{bove-capretta} is traditionally the {\em de-facto} way to define 
recursive programs that include complex argument relations in Type Theory: For each recursive
definition the user introduces an indexed type family with each constructor corresponding
to a particular recursion pattern and indices that correspond to the program variables. The 
function can then be defined by induction on this witness and dependent pattern matching. 
After-the-fact, the programmer can provide such an inductive witness at the call-sites, to 
justify the totality of their definitions. 
Krauss~\cite{krauss-function} proposes a related technique for showing automatically 
the termination of Isabelle functions by extracting their {\em inductive graph} 
and using an induction principle on that graph (that graph roughly corresponds
to the Bove-Capretta inductive witness).

Chargu\'eraud~\cite{chargueraud-10-fix} presented recently a fixpoint combinator which 
uses, internally, extra measure arguments but hides them from the programmer. 
Chargu\'eraud, in an impressive development, uses previous work on recursion theory 
(the work on ``optimal fixpoints'') to fully separate the definition of a recursive 
function from the termination argument. The result is a beautifully engineered library
that has been used to show many difficult examples from previous work. In spirit, Chargu\'eraud's 
technique is closer to well-founded relations and measures than AF relations, and it'd be 
interesting to explore whether some of his ideas for the separation of 
code and termination arguments can be reused in our development. 

Focusing on practicality, Megacz~\cite{Megacz:coinductive-monad} presents an extremely 
pleasant to the user monadic way to structure recursive definitions based on a coinductive 
datatype, which allows one to prove that the recursive definition will terminate 
after-the-fact. To our surprise, Megacz' method is rarely cited in related work.

\section{Directions for future work}\label{sect:future} 

We have already discussed several possibilities for future work. 
Some important directions are the design of more induction principles, such as more general and convenient 
versions of AF mutual induction principles, and ways to bring our development closer to size-change 
termination. 

Although we have identified an appealing and relatively unexplored induction principle, 
further investigation is required to make our approach more practical.
Related work has identified several requirements that have to be met before a 
termination library or methodology can be deemed effective or successful. It has to:
(i) allow one to define complex recursion patterns naturally 
(preferrably by giving code but deferring proof obligations), (ii)~allow computation 
with recursive definitions, (iii) provide unfolding theorems that unfold a recursive function 
at will, and (iv)~provide the ability to reason about recursive functions using induction. 
In the context of a programming language that requires totality checking, the first 
two requirements matter the most but for a proof assistant the latter two are 
equally important and we plan to investigate these reasoning principles in future work. 

Another ambitious direction is tool support and automation, as well as the integration 
of our framework in a practical dependently typed language or proof assistant. 
The biggest challenge there is to develop a methodology so that a tool (be it an interactive environment, 
or an type checker) may give feedback to the programmer to help him synthesize the right termination 
argument, or automatically discharge the various relation inclusion obligations. We are optimistic about this 
direction, because of (i) the tremendous recent progress in SMT solvers and automated reachability checking 
and (ii) the composability of AF relations. Research 
languages such as Trellys\footnote{\url{http://code.google.com/p/trellys/}} and Agda~\cite{norell:thesis}
could potentially provide good candidates for this kind of tool integration.

%% \paragraph{Acknowledgements} This work started from 
%% discussions with Byron Cook and Ranjit Jhala, as well as Simon Peyton Jones, 
%% Max Bolingbroke, and John Hughes. Thanks to Georges Gonthier for several Coq 
%% tips and tricks.

\bibliography{local}


\end{document} 


